    \section{Licznik impulsów}
        \subsection{Opis}
            Licznik impulsów (Pulse Counter) jest peryferium, które znakomicie ułatwia obsługę enkoderów. Procesory bez takich udogodnień są skazane na odczytywanie przerwania za każdym razem jak enkoder wyśle impuls, aby następnie za pomocą kodu Graya rozpoznawać kierunek obrotu \cite{gray}. W takim zastosowaniu użycie enkodera jako pokrętła jest zrozumiałe. Zmieniając enkoder ręczny na taki zamontowany na silniku znacznie zwiększamy częstość generowania impulsów, a co za tym idzie również przerwań, odbieranych w sekundzie pracy procesora. W przypadku użytego w tym projekcie silnika prędkość obrotowa silnika przed przekładnią równa jest:
            
            \begin{displaymath}
              V = \frac{ n \cdot k}{t_{n}} = \frac{ 300 \cdot 20.4 }{60s} = 102 RPS
            \end{displaymath}
            
            Gdzie:
            \begin{itemize}
                \item $n$ - ilość obrotów silnika za przekładnią w ciągu minuty pracy,
                \item $k$ - współczynnik redukcji przekładni,
                \item $t_{n}$ - okres czasu z którego brane są impulsy.
            \end{itemize}
            
            Wiedząc że zintegrowany enkoder wytwarza 11 impulsów na każdy obrót, nietrudno jest wyliczyć że procesor zostanie zasypany ilością 1122 przerwań w ciągu zaledwie jednej sekundy pracy. Taka kolej rzeczy z pewnością odbiłaby się na wydajności urządzenia. W ekstremalnym przypadku użycie większego silnika lub dokładniejszego enkodera mogłoby doprowadzić do sytuacji, gdzie procesor nie byłby w stanie obsłużyć takiego natłoku przerwań. Rozwiązaniem tego problemu jest użycie liczników impulsów, które odciążają mikrokontrolery z takich wyzwań. Posiadają one swoją pamięć w której gromadzą dane na temat zliczonych sygnałów. W takiej sytuacji zadaniem procesora jest jedynie czytanie z tej pamięci. ESP32 posiada aż 8 takich liczników, co stwarza ogromne możliwość wykorzystania tego układu.
            
            Producent układu w dokumentacji procesora informuje że minimalny okres pomiędzy impulsami, który gwarantuje nie pominięcia żadnego odczytu nie może być mniejszy jak $12.5ns$. Wykorzystany w projekcie silnik obraca się z prędkością 100 obrotów na sekundę, a przy każdym obrocie enkoder produkuje 11 impulsów. Częstotliwość występowania impulsów jest następująca:
            
            \begin{displaymath}
              f = \frac{ n }{t_{n}} = \frac{ 100 \cdot 11 }{1s} = 1100Hz
            \end{displaymath}
            
            Gdzie $n$ oznacza generowaną ilość impulsów przez enkoder w okresie czasu $t_{n}$. Okres pomiędzy impulsami jest odwrotnością częstotliwości:
            
            \begin{displaymath}
              T = \frac{1}{ f } = \frac{1}{ 1100Hz } = 909.09 us
            \end{displaymath}
            
            Obliczenia zostały poparte pomiarami z oscyloskopu. Pomiar został umieszczony na rys. \ref{fig:pcnt_osc}. Widać na nim że częstotliwość waha się w okolicy 1040Hz. Delikatne zwiększenie napięcia zasilania spowodowałoby minimalny wzrost obrotów, więc uzyskanie wyliczonej częstotliwości nie jest problemem. Niemniej jednak, okres pomiędzy impulsami enkodera jest o rzędy wielkości większy niż okres graniczny dla tego licznika. Oznacza to że można wypełni ufać wskazaniom licznika ponieważ że żadne impulsy nie są pomijane.
            Abstrahując od okresów pomiędzy impulsami obraz z oscyloskopu wspaniale prezentuje przesunięcie przebiegów pomiędzy kanałami enkodera. 
            
            
            \begin{figure}[ht]
                \centering
                \includegraphics[width=0.9\textwidth]{img/counter_period.png}
                \caption{Pomiar impulsów z oscyloskopu}
                \label{fig:pcnt_osc}
            \end{figure}
            
            
            
            
        \subsection{Konfiguracja}
            W przypadku tego układu, konfiguracja jest bardzo prosta i sprowadza się do wypełnienia zaledwie dwóch struktur i wysłania ich do peryferium.
            
            \begin{kod}
              \inputminted[firstline=5,lastline=48]{cpp}{esp/listings/encoder_driver.cpp}
              \caption{Konfiguracja licznika impulsów}
              \label{code:pcnt}
              \vspace{2em}
            \end{kod} 
            
            Użycie jednej struktury powoduje włączenie jednego z dwóch kanałów w liczniku. Licznik automatycznie ewaluuje zbocza narastające na obu wyjściach enkodera, co podwaja wykrywaną ilość impulsów. Poprawne skonfigurowanie obu kanałów w liczniku w sposób pokazany na listingu \ref{code:pcnt} sprawia że licznik poprzez analizę zbocz narastających jak i opadających zwiększa dokładność enkodera aż czterokrotnie. Ten sprytny zabieg sprawił że użyty w projekcie tani enkoder dostarcza nam aż 897 impulsów na każdy obrót wału wyjściowego z przekładni. Skutkuje to impulsem co około 0.4°. Proces ten został opisany w publikacji \cite{enkoder}. Poza konfiguracją licznika musimy dbać jedynie o odczyt zliczonych impulsów oraz zerowanie pamięci po wykonanym odczycie. Funkcje odpowiedzialne za te czynności zostały przedstawione w listingu \ref{code:pid1}