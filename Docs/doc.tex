\documentclass[a4paper, 12pt]{article}


\usepackage{hyperref}
\usepackage{polski}
\usepackage[utf8]{inputenc}
\usepackage{graphicx}
\usepackage{latexsym}
\usepackage{afterpage}


\begin{document}
  
\begin{titlepage}
  \begin{center}

    \begin{figure}[ht]
      \includegraphics[width=1\textwidth]{img/pwr.png}
    \end{figure}

    \vspace{4cm}
    \textbf{\huge Marcin Bober}
    \vspace{0.5cm}

    Projekt systemu sensorycznego bazującego na protokole MQTT
    \vspace{0.5cm}

    \textbf{\Large Praca dyplomowa inżynierska}
    

    \vspace{7cm}

    \begin{flushleft}
      Prowadzący pracę: \\
      Dr inż. Mateusz Cholewiński
    \end{flushleft}


    \vfill
    Wrocław, 2021
    

  \end{center}
\end{titlepage}



  \leavevmode\thispagestyle{empty}\newpage %pusta strona


  \tableofcontents
  
  \section{Wprowadzenie}
  
  Celem pracy jest stworzenie systemu sensorycznego 
  potrafiącego zarządzać silnikiem prądu stałego.
  

  \section{Założenia}

  System składa się z urządzenia wykonawczego serwera 
  oraz aplikacji dostępowej. 
  Komunikacja pomiędzy elementami systemu 
  następuje poprzez serwer z wykorzystaniem protokołu MQTT.
  
  Urządzenie wykonawcze składa się z mikrokontrolera,
  mostka H oraz silnika prądu stałego wyposarzonego w enkoder
  kwadraturowy. Oprócz tego posiada układ pomiaru napięcia, a
  także wszystkie elementy niezbędne do prawidłowej pracy 
  procesora.

  Aplikacja dostępowa służy do łaczenia się z serwerem 
  poprzez protokół MQTT. Ma za zadanie stanowić prosty i 
  przejrzysty interfejs do sterowania urządzeniem.
  Za jej pomocą mamy możliwość:
  
  \begin{itemize}
    \item zadawania prędkości obrotowej silnika,
    \item odczytu aktualnej prędkości silnika,
    \item zmiany nastaw regulatora PID,
    \item odczytu napięcia zasilania urządzenia,
    \item odczytu wypełnienia sygnału PWM.
  \end{itemize}
  
  Ostatnim elementem systemu jest serwer hostujący broker
  komunikatów w protokole MQTT. Ma on za zadanie być
  pośrednikiem w komunikacji między aplikacją dostępową,
  a urządzeniem wykonawczym.

  \section{Opis komponentów}

  \subsection{Aplikacja dostępowa}
  W celu komfortowej obsługi sterownika silnika,
  zdecydowałem się na stworzenie aplikacji dostępowej w 
  języku C++, która będzie komunikowała się w protokole MQTT.
  W tym celu wykorzystałem otwarto źródłową wersję narzędzia
  programistyczne jakim jest QT.
  Moja decyzja była uwarunkowana moją szeroką znajomością
  tego narzędzia. Ponadto biblioteki QT posiadają dobrze 
  zefiniowane warstwy abstrakcji oraz bardzo wylewną 
  dokumentację, co sprawia że tworzenie zawansowanych, 
  przenośnych aplikacji okienkowych jest wybitnie proste i 
  przyjemne. 

  \subsection{Urządzenie wykonawcze}
  Urządzenie końcowe zostało oparte na wybitnie popularnym
  SoC ESP32-S, chińskiej firmy Espressif Systems. 
  Został on wybrany ze względu na wbudowany moduł WiFi oraz
  bardzo niską cene wynoszącą poniżej 2\$. Ponadto oferuje on:

  \begin{itemize}
    \item dwurdzeniowy procesor o taktowaniu do 240MHz,
    \item 320 KiB RAM, 448 KiB ROM,
    \item Bluetooth w standardzie 4.2 oraz BLE,
    \item 34 wyprowadzenia GPIO.
  \end{itemize}

  Można zauważyć że posiada on dość imponujące parametry 
  jak na tak tani produkt. Pozwoli to na zadowalającą wydajność
  całego systemu bez obaw o niewystarczające zasoby lub 
  kiepską optymalizację.

  W roli mostka H zastosowałem bardzo popularny układ scalony
  L298N w obudowie Multiwatt15. Nie jest to najdoskonalszy wybór,
  ale zdecydowanie wystarczający dla tego zastosowania. 
  Może się on poszczycić napięciem zasilania silników do 46V i 
  prądem szczytowym na poziomie 2A. Ma on dwa kanały co sprawia że
  może jednocześnie sterować dwoma osobnymi silnikami, jednakże
  mój projekt zakłada wykorzystanie tylko jednego silnika, więc 
  drugi kanał pozostanie nieaktywny.

  Silnik ...

  \subsection{Serwer}


  \section{Podsumowane}


  \cite{test}

  \section{Literatura}

  \begin{thebibliography}{99}
    \bibitem{test} H.~Partl: \emph{German \TeX}, TUGboat Vol.~9,, No.~1 ('88)
  \end{thebibliography}

\end{document}