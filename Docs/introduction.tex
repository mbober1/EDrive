  \chapter{Wprowadzenie}
  
    Rozwój technologii powoduje zmiany w każdej dziedzinie życia. Dotyka to nie tylko nasze codzienne otoczenie, ale i przede wszystkim gałęzie przemysłu. To właśnie między innymi potrzeby przemysłu napędzają innowacje poprzez wciąż rosnące zapotrzebowanie na nowe, lepsze i wydajniejsze rozwiązania. 
    
    Dzisiejsza technologia pozwala na bardzo precyzyjne sterowanie takim silnikami prądu stałego, a same sterowniki nie zajmują już połowy pokoju. Wręcz przeciwnie - technika analogowa coraz to częściej musi ustępować tej mikroprocesorowej. Powodów takiego stanu rzeczy jest bardzo wiele. Od większej uniwersalności na łatwość poprawy ewentualnych błędów kończąc. 
    
    Jeżeli chodzi jednak o przemysł i produkcję urządzeń na wielką skalę to jest jeden parametr który przyćmiewa wszystkie inne. Są to oczywiście koszty produkcji. Koszty zaprojektowania i wdrożenia produktu na rynek są również ważne, ale to koszty produkcji są powodem dla którego księgowi rwą po nocach włosy z głowy szukając oszczędności na każdym drobnym elemencie. W tym miejscu pojawiają się mikroprocesory. Układy o bardzo dużej wszechstronności, które można w każdej chwili przeprogramować całkowicie zmieniając ich działanie. W rękach sprawnego programisty są bardzo wydajne, a jednocześnie niezwykle energooszczędne. Jednakże w mojej pracy energooszczędność nie jest najważniejszym celem. 
    
    \section{Cel pracy}
        Myślą przewodnią stojącą za powstaniem tego projektu jest zaprojektowanie systemu zdolnego do sterowania pracą silnika prądu stałego. Co więcej, sterowanie odbywać się będzie w sposób zdalny. Jeden sygnał z komputera i sterownik umieszczony nawet na innej półkuli wysteruje silnik tak, jak sobie tego zażyczymy. Połączenie będzie zrealizowane z wykorzystaniem, bardzo popularnego w robotyce i rozwiązaniach internetu rzeczy, protokołu MQTT. Projekt dopełniać będzie miła dla oka aplikacja okienkowa, która pozwoli na łatwą obsługę i przejrzysty wgląd w najważniejsze parametry pracy silnika. 
        
        Dodatkowym celem dla projektu jest zachowanie jak najniższej ceny, co będzie później warunkowało wybór konkretnych elementów systemu. Ma to również wpływ na poziom trudności projektu, w szczególności regulatora napięcia elementu wykonawczego, ponieważ tanie elementy cechują znacznie gorsze parametry aniżeli drogie, markowe produkty.
    

 
