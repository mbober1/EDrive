  \chapter{Wprowadzenie}
  
    Rozwój technologii powoduje zmiany w każdej dziedzinie życia. Dotyka to nie tylko nasze codzienne otoczenie, ale i przede wszystkim gałęzie przemysłu. To właśnie między innymi potrzeby przemysłu napędzają innowacje poprzez wciąż rosnące zapotrzebowanie na nowe, lepsze i wydajniejsze rozwiązania. Coraz to szybsze procesory oraz miniaturyzacja elektroniki sprawia że niegdyś ogromne i skomplikowane sterowniki silników odchodzą w zapomnienie. 
    
    Dzisiejsza technologia pozwala na bardzo precyzyjne sterowanie takim silnikami prądu stałego, a same sterowniki nie zajmują już połowy pokoju. Wręcz przeciwnie. Technika analogowa coraz to częściej musi ustępować tej mikroprocesorowej. Powodów takiego stanu rzeczy jest bardzo wiele. Od większej uniwersalności na łatwość poprawy ewentualnych błędów kończąc. Jeżeli chodzi jednak o przemysł i produkcję urządzeń na wielką skalę to jest jeden parametr który przyćmiewa wszystkie inne. Są to oczywiście koszty produkcji. Koszty zaprojektowania i wdrożenia produktu na rynek są również ważne, ale to koszty produkcji są powodem dla którego księgowi rwą po nocach włosy z głowy szukając oszczędności na każdym drobnym elemencie. W tym miejscu pojawiają się mikroprocesory. Układy o bardzo dużej wszechstronności które można w każdej chwili przeprogramować całkowicie zmieniając ich działanie. W rękach sprawnego programisty są bardzo wydajne, a jednocześnie niezwykle energooszczędne. Jednakże w mojej pracy energooszczędność nie jest najważniejszym celem. 
    
    Myślą przewodnią stojącą za powstaniem tego projektu jest zaprojektowanie systemu zdolnego do sterowania pracą silnika prądu stałego. Co więcej, sterowanie odbywać się będzie w sposób zdalny. Jeden sygnał z komputera i sterownik umieszczony nawet na innej półkuli w mgnieniu oka wysteruje silnik tak, jak sobie tego zażyczymy. Projekt dopełniać będzie miła dla oka aplikacja okienkowa, która pozwoli na łatwą obsługę i przejrzysty wgląd w najważniejsze parametry pracy silnika. 
    
    Dodatkowym celem dla projektu jest zachowanie jak najniższej ceny, co będzie później warunkowało wybór konkretnych elementów systemu. Ma to również wpływ na poziom trudności projektu, w szczególności regulatora napięcia elementu wykonawczego, ponieważ tanie elementy cechują znacznie gorsze parametry aniżeli drogie, markowe produkty.
    

    % dopisać 1-2 strony lania wody 
  
    \newpage
    \section{Założenia}


    
    Zostały zdefiniowane ogólne założenia dotyczące każdego z elementów, których spełnienie definiuje kryterium sukcesu.
    
    \begin{itemize}
        \item    System składa się z trzech głównych elementów, połączonych ze sobą:
            \begin{itemize}
                \item urządzenia wykonawczego,
                \item brokera MQTT,
                \item aplikacji dostępowej.
            \end{itemize}
            
        \item     Komunikacja pomiędzy częściami składowymi odbywa się z poprzez sieć bezprzewodową wykonaną w technologii WiFi z wykorzystaniem protokołu MQTT.
        \item     Urządzenie wykonawcze ma zadawać napięcia na silnik prądu stałego w taki sposób, aby uzyskać parametry jak najbardziej zbliżone do zadanych przez aplikację dostępową.   
        \item    Aplikacja dostępowa ma za zadanie stanowić prosty i przejrzysty interfejs do sterowania urządzeniem. Umożliwia ona:
    
            \begin{itemize}
                \item zadawania prędkości obrotowej silnika,
                \item odczytu aktualnej prędkości silnika,
                \item zmiany nastaw regulatora PID,
                \item odczytu napięcia zasilania urządzenia,
                \item odczytu wypełnienia sygnału PWM.
            \end{itemize}
            
        \item    Broker MQTT on za zadanie być lekkim i szybkim pośrednikiem w komunikacji między aplikacją dostępową, a urządzeniem wykonawczym.
    \end{itemize}
 

 

    \section{Zgrubny opis komponentów}

      \subsection{Aplikacja dostępowa}
        W celu komfortowej obsługi sterownika silnika, została stworzona aplikacja dostępowa w języku C++, która będzie komunikowała się w protokole MQTT.
        W tym celu wykorzystano otwarto źródłową wersję narzędzia programistycznego jakim jest QT \cite{qt}. Biblioteki QT posiadają dobrze zdefiniowane warstwy abstrakcji oraz bardzo wylewną dokumentację, co sprawia że tworzenie zaawansowanych, przenośnych aplikacji okienkowych jest niezwykle łatwe i przyjemne. 

      \subsection{Urządzenie wykonawcze}
        Urządzenie końcowe jest dedykowanym rozwiązaniem przygotowanym specjalnie na poczet tego projektu. Bazuje ono na układzie z rodziny ESP32. Łączy się ono do brokera MQTT. Znajdujące się tam dane wykorzystuje w procesie sterowania silnikiem, samemu publikując aktualny stan silnika. 
   % TODO dopisać coś jeszcze


      \subsection{Serwer}
        Ostatnim elementem projektu jest Broker MQTT. Jego zadaniem jest bycie łącznikiem pomiędzy urządzeniami. Pozwala on na łatwe subskrybowanie jak i publikowanie informacji. Ta rola przypadła dla otwarto źródłowego brokera Mosquitto wydanego przez fundację Eclipse. Jest to jeden z popularniejszych programów tego typu, a zawdzięcza to małym wymaganiom sprzętowym, skalowalności oraz
        dostępności na wielu architekturach sprzętowych. Co więcej, w celu wdrożenia
        przenośności projektu, oprogramowanie to zostało poddane konteneryzacji,
        dzięki czemu można łatwo transportować go i uruchamiać na innych komputerach wraz z całą konfiguracją przy użyciu programu Docker.