   
\chapter{Szczegółowy opis aplikacji dostępowej}
    \section{Grupa docelowa}
        Aplikacja okienkowa została stworzona, aby zwiększyć dostępność systemu dla osób nie wprawionych w tematy związane z protokołem MQTT. Istnieje pełen przekrój uniwersalnych aplikacji zdolnych do komunikacji za pośrednictwem owego protokołu. Część z nich świetnie naddawałaby się do kooperowania z resztą przygotowanego systemu. Z drugiej strony wszystkie tego typu aplikacje są zazwyczaj nad wyraz rozbudowane oraz wymagają od użytkownika pewnej specyficznej wiedzy i doświadczenia. Nie można wykluczać że z tego rozwiązania miałaby ochotę skorzystać osoba o niższej świadomości technicznej lub chociażby niewdrożona w tematy związane z tą technologią. Z tego też powodu w ramach projektu powstała aplikacja dedykowana opisywanemu systemowi. Gwarantuje ona kompatybilność z resztą elementów zawartych w projekcie jednocześnie posiadając opcje w pełni wykorzystujące wszystkie funkcjonalności systemu. Należy także dodać że projektowana była z myślą o prostocie, intuicyjności w obsłudze i minimalizmie.
        
    \section{Wykorzystana technologia}
        Sprostanie założeniom projektu jest nietrywialnym, a wręcz niedorzecznie trudnym zadaniem bazując jedynie na standardowych bibliotekach. Wykonanie tak zaawansowanej aplikacji w krótkim terminie było jednoznaczne z wykorzystaniem wysokopoziomowego frameworka, który znacząco uprościłby rozwój oprogramowania. W tym celu został użyty  nowoczesny framework QT w wersji otwarto źródłowej. Jest to niezwykle rozbudowane narzędzie do tworzenia aplikacji okienkowych. Wspiera ono pełen przekrój systemów operacyjnych oraz architektur sprzętowych dzięki czemu program opierający się o tą technologię może być z powodzeniem przenoszony na różne urządzenia. Zdecydowanie najciekawszym aspektem tego oprogramowania jest niecodzienny system sygnałów i slotów. Według wielu początkujących programistów jest on nieintuicyjny. Jednakże wraz z korzystaniem z tego rozwiązania i towarzyszącym temu rosnącym doświadczeniem, pogląd ten jest niejednokrotnie rewidowany.  
        
        Poniżej przedstawione jest kilka fragmentów kodu użytego przy budowie aplikacji dostępowej wraz z opisami objaśniającymi wszystkie znajdujące się w nim zawiłości.