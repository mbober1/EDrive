\section{Wykresy}
    \subsection{Definicja klasy}
        Jedną z najważniejszych funkcjonalności aplikacji jest estetyczna prezentacja danych pobranych ze sterownika. Podjęta została decyzja o wykorzystaniu w tym celu prostych wykresów. Framework QT zapewnia moduł QtCharts, który jest zestawem prostych w użyciu komponentów graficznych niezbędnych do stworzenia estetycznych wykresów. Zostało to uskutecznione poprzez wykonanie nowej klasy dziedziczącej po klasie QChart dostępnej w opisywanym module. W swoim konstruktorze tworzy ona wcześniej zdefiniowany wykres z kilkoma seriami danych. Oprócz tego stworzona klasa posiada zdefiniowane metody do czyszczenia zawartości wykresu, dodawania nowych punktów oraz ukrywania nieużywanych serii danych. Kod jest dostępnym w listingu \ref{code:app_chart_class}.
   
        \begin{kod}
          \inputminted[firstline=17]{cpp}{app/listings/chart.hpp}
          \caption{Definicja klasy wykresów}
          \label{code:app_chart_class}
        %   \vspace{1em}
        \end{kod}     
        
    \subsection{Dodawanie punktów do wykresu}
        Implementacja rysowania wykresów opublikowana przez twórców frameworka QT pozostawania wiele do życzenia. W szczególności kwestia wydajności nie została należycie rozpatrzona czego skutkiem jest niewymiernie wysokie zużycie zasobów komputera podczas renderowania tych obiektów. Zastosowana strategia przeciwdziałająca temu zjawisku zakłada skończoną liczbę punktów obecnych jednocześnie na wykresie. Z tego też powodu został zaimplementowany trywialny algorytm pozbywający się nadmiarowych obiektów. 
        
        Kod przedstawiony w listingu \ref{code:app_chart_add} ma za zadanie włączać kolejne dane do wykresu. Uprzednio musi zostać wyznaczona pozycja punktu na osi X. Następnie dla każdej z 3 serii sprawdzana jest ilość znajdujących się na niej pomiarów. W razie przekroczenia ustalonego limitu, nadmiar elementów jest usuwany z początku kolejki, aby następnie dodać na jej końcu dodać nowy punkt. Przed zakończeniem tej metody wykonywana jest kalkulacja położenia najbardziej oddalonych od siebie obiektów, aby móc dopasować zakresy wyświetlania wykresu.
        
        \begin{kod}
          \inputminted[firstline=71, lastline=94]{cpp}{app/listings/chart.cpp}
          \caption{Dodawanie danych do wykresu}
          \label{code:app_chart_add}
        %   \vspace{2em}
        \end{kod}   
        
    \subsection{Czyszczenie wykresu}
        Usuwanie danych w wykresów nie należy do najbardziej skomplikowanych. Wystarczy jedynie na każdej z serii wywołać metodę pozbywającą się wszystkich punktów z bufora. Kod znajduje się w listingu \ref{code:app_chart_remove}.
      
        \begin{kod}
          \inputminted[firstline=97, lastline=105]{cpp}{app/listings/chart.cpp}
          \caption{Usuwanie wszystkich danych z wykresu}
          \label{code:app_chart_remove}
        %   \vspace{2em}
        \end{kod}   
        
    \subsection{Ukrywanie serii}
        Metoda zmiany widoczności danej serii jest jedynie makrem korzystającym z odziedziczonych metod. Dostępna jest pod listingiem \ref{code:app_chart_show}.

        \begin{kod}
          \inputminted[firstline=108]{cpp}{app/listings/chart.cpp}
          \caption{Zmiana widoczności serii}
          \label{code:app_chart_show}
        %   \vspace{2em}
        \end{kod}   